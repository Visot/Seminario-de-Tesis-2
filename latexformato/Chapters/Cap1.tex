\chapter{Introducción}
%-- 10 lineas
Dentro de la inteligencia artificial, las redes neuronales profundas desempeñan un papel muy importante, debido a que estas permiten entrenar a las computadoras para que realicen tareas que nuestros cerebros realizan de manera natural como el reconocimiento de voz, imágenes y patrones. Una características de las redes neuronales profundas es la gran cantidad de capas que poseen. Esto permite que las redes sean capaces de extraer características de los datos ya sean imágenes o voz.\\
 La comunicación entre los seres humanos se realiza por medio del habla, la cual es emitida como una señal de voz debido a la cantidad de información que esta señal posee ha permitido el estudio y desarrollo de aplicaciones como identificadores biométricos, conversión de voz en texto, etc. Tareas como estas requieren un análisis complejo debido a que se necesita tratar problemas como la reverberación y el ruido presentes en el entorno.




\section{Motivación}

La inteligencia artificial constituye una base muy importante en el campo de la computación, esta mezcla un conjunto de disciplinas como la estadística y ciencia de la computación con el objetivo de construir modelos que puedan permitir a las computadoras realizar tareas que hace algunos años hubiesen sido consideradas imposibles. El avance de la inteligencia artificial a permitido el desarrollo de programas que sean capaces de realizar tareas sin haber sido explícitamente programadas para hacerlas.\\
Entre los modelos existentes en la inteligencia artificial las redes neuronales han tenido un amplio desarrollo en las últimas décadas debido que sido utilizada para trata datos más complejos como las imágenes o señales de voz. El tratamiento de la voz es una tarea más compleja que ha sido estudiada en los últimas décadas debido a sus aplicaciones en la identificación biométrica, robótica y procesamiento de lenguaje natural.\\
Actualmente, los sistemas de reconocimiento de voz están presentes en distintos software muchos de ellos permiten romper algunas barreras presentes como por ejemplo para personas con alguna discapacidad física existen software que permiten a las personas ciegas leer la pantalla de su computador transformando el texto en voz otro ejemplo son las personas sordas existen software capaces de transformar voz en texto.\\
Dentro del reconocimiento de voz existen problemas presentes como el ruido de voz y reverberación, fenómeno sonoro producido por la reflexión,  los cuales deben considerados para que una red neuronal sea considera robusta ante estos problemas.\\ Además, debido a la gran cantidad de información que la señal de voz transmite es importante utilizar las herramientas adecuadas para reducir costo computacional ya sea mediante el uso de algoritmos eficientes o hardware más potentes.\\
En la actualidad existen APIs para el reconocimiento de voz pero estás siempre poseen limitaciones por lo cual en esta investigación se busca desarrollar un sistema de reconocimiento de voz que nos permita reconocer el habla y resuelva los problemas ya mencionados, es decir que sea robusta, esto nos permitirá entender y comprender como funciona el lenguaje humano. Este nuevo conocimiento nos permitirá desarrollar una red neuronal capaz de reconocer palabras del lenguaje español en este seminario con un conjunto de datos de números.
%-----
%-----Que es lo que te ha motivado para realizar la tesis en esta temática y los aspectos más esenciales que pensabas obtener de ella...

%-----Este punto podrá ser de 4 páginas máximo. Es una de las partes más importantes que introduce al lector en el trabajo en tu tesis, por lo que debe estar muy bien redactado y estructurado.

\section{Objetivos}

El objetivo de este seminario es el diseñar un sistemas capaz de procesar la voz y transformarla en texto mediante el uso de distintos de modelos de redes neuronales recurrentes aplicado a un conjunto de datos del lenguaje español.

Específicamente, los objetivos de este trabajo con respecto al sistema son:

\begin{itemize}
\item[•] Entender el funcionamiento de las redes neuronales profundas.%--OBJETIVO ESPECÍFICO 1.
\item[•] Conocer el proceso involucrado en el habla humana.
\item[•] Estudiar procesamiento de las señales de voz.

\item[•] Diseñar un sistema robusto capaz de reducir los problemas de reverberación y ruido.
\item[•] Mostrar los resultados obtenidos y explicarlos basándonos en la información estudiada.



\end{itemize}

Y los objetivos con respecto a las competencias académicas desplegadas en el trabajo son:
\begin{itemize}
\item[•] Desarrollar un mejor entendimiento de las redes neuronales y sus aplicaciones, para así poder lograr afrontar problemas en el campo de la inteligencia artificial. %--OBJETIVO COMPETENCIA 1.
\item[•] Obtener la capacidad de discriminar entre los algoritmos para tratar las señales de voz.%--OBJETIVO COMPETENCIA 2.
\item[•] Desarrollar el criterio necesario para trabajar con datos de audio.
\item[•] Obtener un conocimiento de las herramientas y recursos que existen actualmente para abordar problemas de aprendizaje profundo, además de poder analizar que herramientas son adecuadas para algunos problemas.

\end{itemize}

\section{Estructura del Seminario}


\begin{itemize}

\item \textbf{Introducción:} \\
En este capítulo introductorio se comenta sobre el tema a tratar, las motivaciones, intereses, objetivos con los cuales se planteo el presente seminario.
%--En este capítulo introductorio se comenta sobre las motivaciones ...

\item \textbf{Estado del Arte:} \\
Este capítulo muestra los trabajos e investigaciones ya realizadas, además de algunas aplicaciones que motivaron al presente seminario y además las investigaciones mostrarán el interés del problema planteado.

\item \textbf{Redes Neuronales:} \\
En este capítulo daremos una introducción a las redes neuronales de manera general, además veremos los tipos de redes existentes y nos enfocaremos más en las redes neuronales recurrentes.
\item \textbf{Procesamiento de la señal de voz} \\
En este capítulo conoceremos más el proceso del ingreso de la señal a nuestra red como. También describiremos los algoritmos usados en este proceso.
\item \textbf{Resultados:} \\
Se mostrarán los resultados obtenidos en las pruebas de los optimizadores además de describir los resultados.
\item \textbf{Conclusiones y Trabajo Futuro:} \\
En este capítulo se plantean las conclusiones y se detalla algunos inconvenientes encontrados durante el trabajo. Además que se comprueba la teoría descrita en el capítulo 3.
%--\item \textbf{Metodología y Herramientas:} \\
%--....

%--\item \textbf{...} \\
%--Un pequeño texto más de ese capítulo(s) en concreto.

%--\item \textbf{Conclusiones y Trabajo a Futuro:}\\
%--En este capítulo se exponen las conclusiones obtenididas de este trabajo. Adicionalmente, se proponen trabajos a futuro para la implementación de el servidor y la aplicación web del sistema.
\end{itemize}

%--NOTA: RECUERDE QUE ES COMO UN LIBRO TODO CAPÍTULO NUEVO COMENZARÁ CON PÁGINA IMPAR NUNCA PAR


