\chapter{Conclusiones y Trabajo Futuro}
En este capítulo se describirán las conclusiones generales que se encontraron al probar y estudiar las variantes de RNN al momento de realizar una tarea de clasificación en el conjunto de datos utilizado.

Además, se propondrán algunas mejoras para que el trabajo obtenga mejores
resultados en el futuro.

\section{Conclusiones}

\begin{itemize}
	\item El uso 2 capas LSTM y dropout de 0.4 mejoran el rendimiento de nuestro modelo además manejar mejor el problema de overfitting.
	\item Las redes LSTM obtuvieron mejores resultados contra el RNN simple debido a su capacidad de manejar el problema de desaparición de gradiente.
	\item El número de estados ocultos puede acelerar el proceso de entrenamiento pero puede generar efectos de overfitting, por lo cual se obtiene un mejor rendimiento para 128.
	\item El algoritmo MFCC permite extraer las características necesarias para el procesamiento de la señal de voz.
	\item Los audios con distintos tiempos incrementan el número de lapsos tiempos en que desenrolla la red lo cual impide una mejor precisión.
	\item La cantidad de epochs requeridos es menor en las redes con una sola capa LSTM pero estas tienden a caer en problemas de overfitting.	
	
	
\end{itemize}

\section{Trabajo Futuro}

El propósito general de este seminario  fue adquirir el conocimiento y
experiencia necesarios para poder trabajar con datos del tipo dinámico además de conocer las ventajas de usar una red tipo LSTM.
Las LSTM demostraron ser útiles para este tipo de tareas debido a que estas poseen cierta \textit{memoria} y guardan la información pasada. Además que estas mejoran mejor el problema de desaparición de gradiente.
El conjunto de datos recolectados fue suficiente para realizar el entrenamiento y las pruebas pero para un mejor resultado es necesario datos más variados con gran cantidad de hablantes. Los proyecto para generar un corpus de data aún están en desarrollo por lo cual no sé puede acceder a una data más adecuada.\\
En futuros trabajos se trataré de construir sistema de reconocimiento para más palabras además de tratar de construir un lenguaje básico usando LSTM. Este proyecto estuvo delimitada por la capacidad de la tarjeta gráfica y por su conjunto de datos reducido. 

